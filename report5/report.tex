\documentclass[a4paper,11pt]{article}
\usepackage[T1]{fontenc}
\usepackage[utf8]{inputenc}
\usepackage{lmodern}
\usepackage{hyperref}
\usepackage{graphicx}
\usepackage{rotating}
\usepackage{listings}
\usepackage{color}
\usepackage{listings}
\usepackage{pdfpages}

\title{Advanced Algorithms - part 5 (exc 12-13)}
\author{Arash Rouhani (rarash@student.chalmers.se) - 901117-1213}

\begin{document}

\maketitle

\section{"eat one's cake and have it too" combined algorithm}

I propose an algorithm that runs both the algorithms in parallel,
and returns the answer as soon as any algorithm terminates.
We assign different priorities to the algorithms depending on
the parameter $s$.

The first objection might be \emph{why paralell}? Why not just
one of them? Will it not always be faster to just pick one of them?
Yes! But we don't know which one as of the uncertainty introduced
by the randomized algorithm.
Our randomized algorithm
doesn't have a predictable end, if we run the randomized algorithm
on it's own we get no $O(g(n))$ bound. Running only the deterministic
algorithm will not have expected time $s*f(n)$.

So we need both algorithms and also in parallel.
But how do we ensure expected time $s*f(n)$, well
that we obviously get if the "thread" running the randomized
algorithm would have a slowdown of $s$, that is it get
$1/s$ of the attention in the parallel execution.
From this we get that the deterministic "thread" get
$1-1/s$ attention. Here we have assumed
perfect parallelism.

Ok, can we show that this new algorithm both is $O(g(n))$ time and
$s*f(n)$ expected running time? As for the expected running time
we know it is $s*f(n)$ as we choose the $1/s$ factor for
the randomized algorithm just for this reason. Note that if
we have insanely chosen $s$ to be large we evetually
get the purpose-defeating inequality $s*f(n) > g(n)$.
In that case the expected running time would be $g(n)$ or better.
As for if the algorithm is $O(g(n))$. Yes, as the running
time in the worst case is when the deterministic algorithm terminates
before the randomized, yielding running time of

\[
\frac{1}{1-1/s}g(n) = c(s)*g(n) \in O(g(n))
\]

Since $c(s)$ is a constant with respect to $n$ the
algorithm is $O(g(n))$.

\section{Family of colorings}

\textbf{13.1:}
\textbf{13.2:}
\textbf{13.3:}

\end{document}
