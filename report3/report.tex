\documentclass[a4paper,11pt]{article}
\usepackage[T1]{fontenc}
\usepackage[utf8]{inputenc}
\usepackage{lmodern}
\usepackage{hyperref}
\usepackage{graphicx}
\usepackage{rotating}
\usepackage{listings}
\usepackage{color}
\usepackage{listings}
\usepackage{pdfpages}

\title{Advanced Algorithms - part 3 (exc 7-9)}
\author{Arash Rouhani (rarash@student.chalmers.se) - 901117-1213}


\begin{document}

\maketitle

% \includepdf[pages=-]{20111105191330882.pdf} % probably needed later

\section{Bigamical Matching}

Recall the standard flow-solution to bipartite matching.
By making a minor edit to that, we can get it to solve for our problem.
I will first shortly recall the standard solution, then I will
clarify how our problem \emph{and} solution differs.
Lastly I will motivate why that solution solves for our modified version.

In the standard problem we have that each $X$ matches to at most $1$ $Y$.
The solution then is to create graph with a source node $s$ and
sink node $t$. Then connect edges $s \to x$ and $y \to t$. Also
we add the edges from $E$, that is $x \to y$ $(x,y) \in E$,
this time the edges are directed. All edges in this constructed graph
has capacity $1$.
If we solve for maximum flow in this graph, we will get an optimal
solution. $x$ matches to $y$ iff $f_{(x,y)} = 1$.

But the problem statement is that $X$ can match to at most $2$ $Y$.
I claim that to accomodate for this difference, the only modification
is to give capacity $2$ for the $s \to x$ edges.
We still have the exact same interpretation for $f_{(x,y)} = 1$.

Motivation: $0 \leq f_{(s, x)} \leq 2$ and $0 \leq f_{(y, t)} \leq 1$
represents number of matches $x$ and $y$ has respectively,
this matches the problem. $0 \leq f_{(x, y)} \leq 1$ is indeed
a "boolean" for if $x$ matches to $y$. For example for a fixed
$x$, if $f_{(s, x)} = 2$, the rule of \emph{flow conservation}
forces for $x$ to match to two different $y_1$ and $y_2$.
$y_1 \neq y_2$ because that otherwise would contradictory imply
$f_{(x, y_1)} = 2 \leq 1$. Flow conservation
at all node types have a semantic in the problem desrciption.
We just looked at node type $x$. Similar simple analysis
can be done for $s$, any $y$, and $t$ nodes as well,
but we ommit that.

Since this is only a flow problem, this can be solved
in polynomial time.

\section{Dominos on a grid-set}

This is also matching! But between which two "genders"?
We will partition $Q$ into $X,Y \subset Q$.
If we paint the squares on $Q$ like a chess board, then we let
$X$ be the white and $Y$ be the black squares.
The intuition here is that if you put a domino brick on a chess board
it will cover one white and one black square. So if a brick covers
$q_1,q_2 \in Q$, we know that we can write it as
$x=q_1 \in X$ and $y=q_2 \in Y$.

One should now recall the problem's
constraints on brick placement, "two adjacent squares" means one
square is black the other white. "must not overlap" will mean that
we can actually see this as a matching problem, $x$ and $y$ are
matched if they share the same domino brick. In the original
marriage problem a person (male or female) can't marry twice,
here a square (black or white) can't have two domino bricks covering it.

So, we algorithm-wise we use maximum flow time time aswell.
We construct the graph
$G = (X, Y, \{(x,y) | x \in X, y \in Y, adjacent(x, y)\)$
and solve the matching problem. We put a domino on square $x$ and $y$
if $f_{(x,y)} = 1$. We have many times discussed how bipartite
matching is solved, both in class and in the previous problem.

This can run in $O(V^2)$ time, an augmenting path can be found in
$O(E)=O(V)$ and each found path increases flow by $1$, that increment
won't be found more than $V$ times as that would "overflow".
Hopcroft-Karp algorithm for bipartite matching even gives
running time of $O(V^1.5)$.

\section{Multi-source/sink Edge-disjoint connectivity}

We are given $G = (V, E)$, we have some sources $s_1,..,s_k \in V$
and sinks $t_1,..,t_k \in V$. We can solve the problem with maximum
flow. We will construct a flow-graph $G'$ from $G$ and after calculating
the flow in $G'$ we will contruct paths by following flows.
Lets now look at it a bit more in detail.

Construct the graph $G' = (V \cup {S, T}, E \cup E')$.
So $S$ and $T$ are new nodes, the \emph{real} source and sink.
$S$ will provide $s_1..s_k$ with flow.
Looking at $G$, the small sources will have their own
production, while in reality it comes from $S$. $T$ will
in the same sense be a supersink for $t_1..t_k$.
For $S$ and $T$ to really be what I just described we
set $E' = {(S, s_i) | i=1..k} \cup {(t_i, T) | i=1..k}$.
We set that all edges have the capacity $1$.

The maximum flow $m$ from this network will be the number
of paths we can construct. If $m=k$ then we indeed can
find $k$ paths, each path going from $s_i$ to $t_j$.
Since the problem was to find the actual paths and not
any maximize $m$ problem, we consider $m=k$ and
will give an algorithm for finding the paths given the
flow.

Let us say path $i$ is the path starting from $s_i$.


\end{document}
